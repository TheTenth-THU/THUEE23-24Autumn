\documentclass[10pt, a4paper, oneside, fontset=none]{ctexart}
%调用宏包
\usepackage{amsmath, amsthm, amssymb, graphicx, mathrsfs}
\usepackage[bookmarks=true, colorlinks, citecolor=blue, linkcolor=black]{hyperref}
\usepackage{color, framed, geometry, tcolorbox, nicematrix}
\tcbuselibrary{breakable}%box跨页
\tcbuselibrary{skins}%box跨页不留边
\usepackage{makecell, booktabs}
\usepackage[labelfont={bf}]{caption}
\usepackage{multicol}
\usepackage{enumitem}
\usepackage{extarrows}
\usepackage{esint}
\usepackage{yhmath}
\usepackage{dashrule}% 虚线分割线
\usepackage[text=\includegraphics{C:/Users/16870/.vscode/LaTeX_Application/tex/THUEE23-23Autumn/图标简稿.png},angle=0]{draftwatermark}%水印
%\usepackage{tikz}
%基本字体设置
\catcode`\,=\active
\def ,{\textup{,}\hskip0.5em }
\setmonofont{Iosevka}
\setCJKmainfont{FZXSSK.TTF}[BoldFont={SourceHanSerifCN-Bold.otf}, ItalicFont={FZXKTK.TTF}, BoldItalicFont={汉仪颜楷W.ttf}]
\setCJKsansfont{汉仪文黑-45W.ttf}[BoldFont={汉仪文黑-75W.ttf}, ItalicFont={FZYanZQKSJF.TTF}]
\setCJKmonofont{LXGWNeoXiHei.ttf}
%附加字体设置
\newCJKfontfamily{\kaico}{可口可乐在乎体 楷体Coca-ColaCareFontKaiTi.TTF}
\newCJKfontfamily{\kai}{FZXKTK.TTF}[BoldFont={汉仪颜楷W.ttf}, ItalicFont={方正清刻本悦宋 简繁.TTF}, BoldItalicFont={FZYanZQKSJF.TTF}]
\newCJKfontfamily{\yan}{方正清刻本悦宋 简繁.TTF}[ItalicFont={FZYanZQKSJF.TTF}]
\newCJKfontfamily{\xiu}{方正宋刻本秀楷_GBK.TTF}[ItalicFont={方正宋刻本秀楷_GBK.TTF}, BoldFont={FZYanZQKSJF.TTF}]
\newCJKfontfamily{\run}{汉仪润圆-45W.ttf}[BoldFont={汉仪润圆-75W.ttf}, ItalicFont={汉仪润圆-45W.ttf}]
\newCJKfontfamily{\wen}{汉仪文黑-45W.ttf}[BoldFont={汉仪文黑-75W.ttf}, ItalicFont={hk4e_zh-cn.ttf}]
%文档格式
\geometry{left=2.54cm, right=2.54cm, top=3.18cm, bottom=3.18cm}
\setcounter{tocdepth}{3}
\setcounter{secnumdepth}{4}
\linespread{1.4}
\numberwithin{equation}{section}
\renewcommand{\theparagraph}{\Alph{paragraph})}
\newcommand{\Section}[1]{ \refstepcounter{section} \section*{*\thesection\texorpdfstring{\quad}{} #1} \addcontentsline{toc}{section}{\makebox[0pt][r]{*}\thesection\texorpdfstring{\ \ }{} #1} }
\newcommand{\Subsection}[1]{ \refstepcounter{subsection} \subsection*{*\thesubsection\texorpdfstring{\quad}{} #1} \addcontentsline{toc}{subsection}{\makebox[0pt][r]{*}\thesubsection\texorpdfstring{\ \ }{} #1} }
\newcommand{\Subsubsection}[1]{\refstepcounter{subsubsection}\subsubsection*{*\thesubsubsection\texorpdfstring{\quad}{}#1} \addcontentsline{toc}{subsubsection}{\makebox[0pt][r]{*}\thesubsubsection\texorpdfstring{\ \ }{}#1}}
\setlist[itemize]{leftmargin=3em, labelsep=0.25em, itemindent=0em, itemsep=0pt, parsep=0pt, topsep=0pt, partopsep=0pt}
\setlist[description]{leftmargin=4em, labelsep=1em, itemindent=-1em, itemsep=0pt, parsep=0pt, topsep=3pt, partopsep=0pt}
%定理环境
\theoremstyle{plain}
\newtheorem{theorem}{定理}[subsection]
\newtheorem{definition}{定义}[section]
\newtheorem{lemma}[theorem]{引理}
\newtheorem{corollary}[theorem]{推论}
\newtheorem{proposition}[theorem]{命题}

\theoremstyle{definition}
\newtheorem{examplein}[theorem]{\run 例题}
\newtheorem{circum}[theorem]{情形}

\newcommand{\exampleparameter}{0}
\newenvironment{example}[1][0]{% 0/1: no space; 2/3: 5pt space
	\renewcommand{\exampleparameter}{#1}
	\ifnum \exampleparameter>1
		\vspace{10pt}
	\fi
	\hrule
	\vspace{3pt}
	\noindent\hdashrule{\linewidth}{0.5pt}{2pt}
	\vspace{-2em}
	\begin{examplein}
}{% 0/2: -0.5pt space; 1/3: 5pt space
	\end{examplein}
	\vspace{-1em}
	\noindent\hdashrule{\linewidth}{0.5pt}{2pt}\vspace{3pt}
	\hrule
	\ifnum 1=\exampleparameter
		\vspace{10pt}
	\else
		\ifnum 3=\exampleparameter
			\vspace{10pt}
		\else
			\vspace{-0.5pt}
		\fi
	\fi
}

\newenvironment{proofs}[1][\proofname]{\begin{pf}[breakable, enhanced jigsaw]\adjline\begin{proof}[\small#1]\small\kai}{\end{proof}\end{pf}}
\newenvironment{solution}{\begin{proofs}[\small\textit{\yan 解}]\small\renewcommand{\qedsymbol}{$\circledS$}}{\end{proofs}}
\renewcommand{\proofname}{\yan{证明}}
%颜色命名
\definecolor{meihong}{rgb}{0.85,0.2,0.47}
\definecolor{bali}{rgb}{0.2,0.6,0.78}
\definecolor{qinglv}{rgb}{0,0.35,0.32}
%box环境
\newtcolorbox[use counter=definition,number within=section]{defi}[2][]{colback=bali!5!white,colframe=bali!75!black,fonttitle=\sffamily\wen\bfseries,title=重要定义~\thetcbcounter. #2,#1}
\newtcolorbox[use counter=theorem,number within=subsection]{theo}[2][]{colback=meihong!5!white,colframe=meihong!75!black,fonttitle=\sffamily\wen\bfseries,fontupper=\run,title=重要定理~\thetcbcounter. #2,#1}
\newtcolorbox[use counter=definition,number within=section]{defil}[2][]{colback=bali!5!white,colframe=bali!75!black,fonttitle=\sffamily\wen\bfseries,label=#2,title=重要定义~\thetcbcounter. #2,#1}
\newtcolorbox[use counter=theorem,number within=subsection]{theol}[2][]{colback=meihong!5!white,colframe=meihong!75!black,fonttitle=\sffamily\wen\bfseries,fontupper=\run,label=#2,title=重要定理~\thetcbcounter. #2,#1}
\newtcolorbox[auto counter,number within=section]{note}[2][]{colback=qinglv!5!white,colframe=qinglv!75!black,fonttitle=\sffamily\wen\bfseries,title=注~\thetcbcounter. #2,#1}
\newtcolorbox{prenote}[2][]{colback=qinglv!5!white,colframe=qinglv!75!black,fonttitle=\bfseries,title=#2,#1}
\newtcolorbox{pf}[1][]{colback=black!5!white,colframe=white!75!black,#1}
%\newcommand{\mybox}[1]{\tikz[baseline=(MeNode.base)]{\node[rounded corners, fill=gray!20](MeNode){#1};}}

%定义格式记号
\newcommand{\hang}[1][1]{\hangafter 1 \hangindent #1em \noindent}
\newcommand{\adjline}[1][4]{	\lineskiplimit=3pt
	\lineskip=3pt
	\abovedisplayskip=#1pt
	\belowdisplayskip=#1pt}
\newcommand{\den}[2][]{\begin{defi}{#1}\adjline
	\kai #2\end{defi}}
\newcommand{\din}[2][]{\begin{theo}{#1}\adjline
	\run #2\end{theo}}
\newcommand{\de}[2][]{\begin{defil}{#1}\adjline
	\kai #2\end{defil}}
\newcommand{\di}[2][]{\begin{theol}{#1}\adjline
	\run #2\end{theol}}
\newcommand{\dep}[3][]{\begin{defi}{#1\page{#2}}\adjline
	\kai #3\end{defi}}
\newcommand{\dip}[3][]{\begin{theo}{#1\page{#2}}\adjline
	\run #3\end{theo}}
\newcommand{\zhu}[2][]{\begin{note}{#1}\adjline
	\xiu #2\end{note}}
\newcommand{\colors}[1]{\color{#1!75!black}}
\newcommand{\tboba}[1]{\textbf{\kai\color{bali!75!black}#1}}
\newcommand{\mboba}[1]{\kai\boldsymbol{\color{bali!75!black}#1}}
\newcommand{\tbome}[1]{\textbf{\run\color{meihong!75!black}#1}}
\newcommand{\mbome}[1]{\run\boldsymbol{\color{meihong!75!black}#1}}
\newcommand{\tboqi}[1]{\textbf{\xiu\color{qinglv!75!black}#1}}
\newcommand{\mboqi}[1]{\xiu\boldsymbol{\color{qinglv!75!black}#1}}
%定义算符
\def\upint{\mathchoice%
	{\mkern13mu\overline{\vphantom{\intop}\mkern7mu}\mkern-20mu}%
	{\mkern7mu\overline{\vphantom{\intop}\mkern7mu}\mkern-14mu}%
	{\mkern7mu\overline{\vphantom{\intop}\mkern7mu}\mkern-14mu}%
	{\mkern7mu\overline{\vphantom{\intop}\mkern7mu}\mkern-14mu}%
 \int}
\def\lowint{\mkern3mu\underline{\vphantom{\intop}\mkern7mu}\mkern-10mu\int}
\renewcommand{\cfrac}[2]{\genfrac{}{}{}{0}{\raisebox{0.6em}{$#1$}}{\raisebox{-0.8em}{$#2$}}}
\newcommand{\dif}{\mathop{}\!\mathrm{d}}
\newcommand{\e}{\mathrm{e}}
\renewcommand{\i}{\mathrm{i}}
\newcommand{\R}{\mathbb{R}}
\newcommand{\C}{\mathbb{C}}
\renewcommand{\Re}{\mathop{}\!\mathfrak{Re}\mathop{}\!}
\renewcommand{\Im}{\mathop{}\!\mathfrak{Im}\mathop{}\!}
\newcommand{\tp}{^\mathrm{T}}
\renewcommand{\v}[1]{\vec{\boldsymbol{#1}}}
\newcommand{\V}[2][-0.665]{\vec{#2\hspace{#1em}#2\hspace{#1em}#2}}
\newcommand{\dint}[1][]{\displaystyle\int #1}
\newcommand{\page}[1]{\hfill P$_\text{#1}$}
\newcommand{\sumi}[1][n]{\sum\limits_{#1=1}^\infty}
\renewenvironment{cases}[1][l]{\left\{\,\begin{NiceArray}{#1}}{\end{NiceArray}\right.}

%标题、作者、日期
\title
{
	\textbf{复变函数与数理方程}\textit{知识与方法}
}
\author{\fontsize{10.95}{15}\selectfont T$^\text{T}$T}
\date{\fontsize{10.95}{15}\selectfont\kai \today}
%----------------------------------------------------------
\begin{document}

\maketitle
\begin{multicols}{2}
	\begin{flushleft}
		\tableofcontents
	\end{flushleft}
\end{multicols}

\adjline

\newpage
%----------------------------------------------------------
\section{复变函数}

\subsection{复数与复变函数}

\subsubsection{复数的定义与运算}

\begin{definition}
    设 \(z=x+\i y\),其中 \(x\) 和 \(y\) 是实数,\(\i\) 是虚数单位,满足 \(\i^2=-1\),则称 \(z\) 为\tboba{复数},\(x\) 为\tboba{实部},\(y\) 为\tboba{虚部},记作 \(z=\Re z+\i\Im z\)。
\end{definition}

\begin{definition}
    改用极坐标 \(\rho\) 和 \(\varphi\) 表示复数 \(z\),即可得到复数 \(z\) 的\tboba{三角式} \(z=\rho(\cos\varphi+\i\sin\varphi)\) 或\tboba{指数式} \(z=\rho\e^{\i\varphi}\)。其中 \(\rho=|z|=\sqrt{x^2+y^2}\) 为复数 \(z\) 的\tboba{模},\(\varphi=\mathop\mathrm{Arg} z\) 为复数 \(z\) 的\tboba{幅角}。

    约定:以 \(\arg z\) 表示复数 \(z\) 的幅角中满足 \(0 \le \mathop\mathrm{Arg} z < 2\pi\) 的一个特定值,称为 \(z\) 的\tboba{幅角主值}。
\end{definition}

\begin{definition}
    复数 \(z=x+\i y = \rho(\cos\varphi + \i\sin\varphi) = \rho\e^{\i\varphi}\) 的\tboba{共轭}定义为 \(z^* = x-\i y = \rho(\cos\varphi - \i\sin\varphi) = \rho\e^{-\i\varphi}\)。
\end{definition}

复数的运算规则与实数类似,即满足交换律、结合律、分配律等基本运算法则。乘、除、乘方、开方运算使用三角式或指数式更为方便:
\di[复数三角式或指数式的乘、除、乘方、开方运算]{
    设复数 \(z_1=\rho_1(\cos\varphi_1+\i\sin\varphi_1)=\rho_1\e^{\i\varphi_1}\) 和 \(z_2=\rho_2(\cos\varphi_2+\i\sin\varphi_2)=\rho_2\e^{\i\varphi_2}\),则有:
    \begin{align*}
        &z_1z_2 = \rho_1\rho_2\big(\cos(\varphi_1+\varphi_2)+\i\sin(\varphi_1+\varphi_2)\big)
        = \rho_1\rho_2\e^{\i(\varphi_1+\varphi_2)},\\
        &\frac{z_1}{z_2} = \frac{\rho_1}{\rho_2}\big(\cos(\varphi_1-\varphi_2)+\i\sin(\varphi_1-\varphi_2)\big)
        = \frac{\rho_1}{\rho_2}\e^{\i(\varphi_1-\varphi_2)},\\
        &z_1^n = \rho_1^n\big(\cos n\varphi_1+\i\sin n\varphi_1\big)
        = \rho_1^n\e^{\i n\varphi_1},\\
        &\sqrt[n]{z_1} = \sqrt[n]{\rho_1}\left(\cos\frac{\varphi_1+2k\pi}{n}+\i\sin\frac{\varphi_1+2k\pi}{n}\right)
        = \sqrt[n]{\rho_1}\e^{\i\frac{\varphi_1+2k\pi}{n}},\quad k=0,1,2,\cdots,n-1
    \end{align*}
}

\zhu[幅角运算]{
    (1)复数 \(z\) 的幅角 \(\varphi\) 不能唯一确定,其可以加减 \(2k\pi\),即 \(\varphi+2k\pi\) 也是复数 \(z\) 的幅角,其中 \(k=0,\pm1,\pm2,\cdots\)。因此,根式 \(\sqrt[n]{z}\) 的幅角也就可以加减 \(\dfrac{2\pi}{n}\) 的整数倍,从而 \(\sqrt[n]{z}\) 有 \(n\) 个值。
    
    (2)一般地,\(\mathop\mathrm{Arg}(z_1z_2) = \mathop\mathrm{Arg}z_1 + \mathop\mathrm{Arg} z_2\),但 \(\mathop\mathrm{Arg} z^2 = \mathop\mathrm{Arg} z + \mathop\mathrm{Arg} z \neq 2\mathop\mathrm{Arg} z\)。
}

\begin{example}
    将实值坐标平面上的直线方程 \(ax+by+c = 0\) 表示为复数形式。
\begin{solution}
    令直线上 \(z=x+\i y\),则有 \(x=\Re z=\dfrac{z+z^*}{2}\),\(y=\Im z=\dfrac{z-z^*}{2\i}\),因此有
    \[
        a \frac{z+z^*}{2} + b \frac{z-z^*}{2\i} + c = 0 \quad \Rightarrow \quad (a-\i b)z + (a+\i b)z^* + 2\i c = 0
    \]
    即
    \[
        B^* z + B z^* + C = 0
    \]
    其中 \(B \in \C\),\(C \in \R\)。
\end{solution}
\end{example}

\subsubsection{复变函数与解析函数}

\de[复变函数、解析函数]{
    在复平面上存在一个点集 \(E\),若对于每一个 \(z \in E\),都按照一定规律有一个或多个复数 \(w\) 与之对应,则称这种对应关系为从 \(E\) 到复数集 \(\C\) 的\tboba{复变函数},记作 \(w=f(z)\)。其中,\(z\) 称为 \(w\) 的\tboba{宗量},\(E\) 称为 \(f\) 的\tboba{定义域}。

    若函数的定义域 \(E\) 是一个\tboba{区域}:
    \begin{itemize}
        \item \(E\) 是全由\tboba{内点}组成的点集;
        \item \(E\) 具有\tboba{连通性},即对于任意两点 \(z_1\) 和 \(z_2\),连接 \(z_1\) 和 \(z_2\) 的折线段也存在于 \(E\) 中;
    \end{itemize}
    并且在区域上每一点都\tboba{解析}(定义\,\ref{解析}),则称复变函数 \(f\) 为\tboba{解析函数}。
}

典型的复变函数有:
\begin{itemize}
    \item \textbf{多项式}:\(f(z)=a_0+a_1z+a_2z^2+\cdots+a_nz^n\);
    \item \textbf{有理分式}:\(f(z)=\dfrac{P(z)}{Q(z)}\),其中 \(P(z)\) 和 \(Q(z)\) 为多项式函数;
    \item \textbf{根式}:\(f(z)=\sqrt[n]{z-a}\);
    \item 部分\textbf{初等函数}:
    \begin{itemize}[leftmargin=1em]
        \item \textbf{指数函数}:\(f(z)=\e^z=\e^{x+\i y}=\e^x(\cos y+\i \sin y)\);
        \item \textbf{三角函数}:\(\sin z=\dfrac{\e^{\i z}-\e^{-\i z}}{2\i}\),\(\cos z=\dfrac{\e^{\i z}+\e^{-\i z}}{2}\),\\
             这里 \(\sin^2 z + \cos^2 z = 1\) 仍成立,但 \(|\sin z|^2 + |\cos z|^2 \ge 1\);
        \item \textbf{双曲函数}:\(\sinh z=\dfrac{\e^z-\e^{-z}}{2}\),\(\cosh z=\dfrac{\e^z+\e^{-z}}{2}\);
        \item \textbf{对数函数}:\(\ln z=\ln \big(|z|\e^{\i\mathop\mathrm{Arg} z})=\ln|z|+\i\mathop\mathrm{Arg} z\);
        \item \textbf{幂函数}:\(z^\alpha=\e^{\alpha\ln z}\)。
    \end{itemize}
\end{itemize}

\begin{example}[3]
    在\tboba{反演}变换 \(w = \dfrac{1}{z}\) (\(z \neq 0\))中,\(z\) 平面上的下面曲线各映射为 \(w\) 平面上的什么曲线?

    (1)\(|z| = 2\);(2)\(\Re z = 1\)。
\begin{solution}
    设 \(z = x + \i y\),\(w = u + \i v\),则
    \[
        w = \frac{1}{z} = \frac{1}{x+\i y} = \frac{x-\i y}{x^2+y^2} = \frac{x}{x^2+y^2} - \i\frac{y}{x^2+y^2} \quad \Rightarrow \quad u = \frac{x}{x^2+y^2},v = -\frac{y}{x^2+y^2}
    \]

    (1)\(x^2+y^2=4\),则 \(u = \dfrac{x}{4},v = -\dfrac{y}{4}\),即 \(u^2+v^2=\dfrac{1}{16}\),为 \(w\) 平面上的圆心在原点、半径为 \(\dfrac{1}{4}\) 的圆,复平面上的表示为 \(|w| = \dfrac{1}{2}\)。

    (2)\(x=1\),则 \(u = \dfrac{1}{1+y^2},v = -\dfrac{y}{1+y^2}\),即 \(u^2+v^2=\dfrac{1+y^2}{(1+y^2)^2}=\dfrac{1}{1+y^2}=u\),为 \(w\) 平面上的圆心在 \(\left(\dfrac{1}{2},0\right)\)、半径为 \(\dfrac{1}{2}\) 的圆,复平面上的表示为 \(\left|w-\dfrac{1}{2}\right| = \dfrac{1}{2}\)。
\end{solution}
\end{example}

\subsubsection{复变函数的导数}

\de[复变函数的导数]{
    设函数 \( w = f(z) \) 是在区域 \(B\) 上定义的单值函数,若在 \(B\) 内对于任意一点 \(z\),存在极限
    \begin{equation}
        \lim\limits_{\Delta z \to 0} \frac{\Delta w}{\Delta z} = \lim\limits_{\Delta z \to 0} \frac{f(z+\Delta z)-f(z)}{\Delta z}
    \end{equation}
    且其值与 \(\Delta z \to 0\) 的方式无关,则称函数 \(w = f(z)\) 在 \(z\) 处\tboba{可导},称该极限为函数 \(f(z)\) 在点 \(z\) 处的\tboba{导数},记作 \(f'(z)\) 或 \(\dfrac{\dif w}{\dif z}\)。
}

形式上看,复变函数和实变函数导数的定义相同,因此在某些方面
具有一致性。然而,\textbf{它们具有本质的区别,因为实变函数只需要沿
实轴逼近极限,而复变函数则需要在全方向逼近极限。}

下面给出函数可导的必要条件:

\di[Cauchy-Riemann条件]{
    记复变函数 \(w = f(z)\) 中,\(\Re w = u\),\(\Im w = v\),\(\Re z = x\),\(\Im z = y\),\(|z| = \rho\),\(\mathop\mathrm{arg} z = \varphi\),则函数可导的必要条件是各偏导数存在且满足\,\tbome{Cauchy-Riemann方程}:
    \begin{equation}
        \begin{cases}
            \dfrac{\partial u}{\partial x} = \dfrac{\partial v}{\partial y},\\[7pt]
            \dfrac{\partial u}{\partial y} = -\dfrac{\partial v}{\partial x},
        \end{cases}
        \quad \text{或} \quad
        \begin{cases}
            \dfrac{\partial u}{\partial \rho} = \dfrac{1}{\rho}\dfrac{\partial v}{\partial \varphi},\\[7pt]
            \dfrac{1}{\rho}\dfrac{\partial u}{\partial \varphi} = -\dfrac{\partial v}{\partial \rho}
        \end{cases}
    \end{equation}
}

\begin{example}
    设有函数 \(f(z) = \sqrt{\left| \Im z^2 \right|}\),求其在复平面原点的可导性。
\begin{solution}
    设 \(z = x + \i y\),知 \(u = \Re f(z) = \sqrt{|2xy|}\),\(v = \Im f(z) = 0\)。则在\((0,0)\)点有
    \begin{align*}
        & \dfrac{\partial u }{\partial x } = \lim\limits_{\Delta x \to 0} \dfrac{u(\Delta x,0) - u(0,0)}{\Delta x} = 0 = \dfrac{\partial v }{\partial y },\\
        & \dfrac{\partial u }{\partial y } = \lim\limits_{\Delta y \to 0} \dfrac{u(0,\Delta y) - u(0,0)}{\Delta y} = 0 = -\dfrac{\partial v }{\partial x }
    \end{align*}
    因此满足 Cauchy-Riemann 方程。然而,从定义看,由于 \(\dfrac{\Delta f(z)}{\Delta z } = \dfrac{f(\Delta z) - f(0)}{\Delta z} = \dfrac{\sqrt{|2\Delta x\Delta y |}}{\Delta x + \i \Delta y}\),因此 
    \[
        \lim\limits_{\Delta x \to 0^+ \atop \Delta x = \Delta y} \dfrac{\Delta f(z)}{\Delta z } = \dfrac{\sqrt{2}}{1 + \i},\qquad
        \lim\limits_{\Delta x \to 0^+ \atop \Delta y = 0} \dfrac{\Delta f(z)}{\Delta z } = 0
    \]
    这说明函数在原点不可导。\textbf{其原因是:函数\(\boldsymbol{u(x,y)}\)在\(\boldsymbol{(0,0)}\)点不可微。}
\end{solution}
\end{example}

\di[单值复变函数可导的充要条件]{
    单值函数\(f(z) = u(x,y) + \i v(x,y)\)可导,当且仅当 \(u(x,y)\) 和 \(v(x,y)\)可微,并且各偏导数满足定理\,\ref{Cauchy-Riemann条件}\,的 Cauchy-Riemann 条件。
}

\de[解析]{
    若函数 \(f(z)\) 在点 \(z_0\) 及其邻域上处处可导,则称 \(f(z)\) 在点 \(z_0\) 处\tboba{解析}。
    
    若函数 \(f(z)\) 在区域 \(B\) 上的每一点都解析,则称 \(f(z)\) 在区域 \(B\) 上\tboba{解析},此时称 \(f(z)\) 为\tboba{解析函数}。
}

\zhu[解析函数实部与虚部的关系]{
    解析函数的实部和虚部不是独立的,知道了其中之一,例如实
    部\(u\),根据Cauchy-Riemann条件就可以唯一地(相差一个常数)确定虚部\(v\),
    这是因为
    \[
        \dif v = \dfrac{\partial v}{\partial x}\dif x + \dfrac{\partial v}{\partial y}\dif y = -\dfrac{\partial u}{\partial y}\dif x + \dfrac{\partial u}{\partial x}\dif y
    \]
    从而有
    \[
        v = \int \dif v = -\int \dfrac{\partial u}{\partial y}\dif x + \int \dfrac{\partial u}{\partial x}\dif y + C
    \]
}

\begin{theorem}
    若函数 \(f(z) = u(x,y) + \i v(x,y)\) 在区域 \(B\) 上解析,则\\
    (1)\(u = \Re f(z)\),\(v = \Im f(z)\)均为 \(B\) 上的\tboba{调和函数}:
    \begin{itemize}
        \item \(u\),\(v\)在区域 \(B\) 上均有二阶连续偏导数;
        \item \(u\),\(v\)均满足\tboba{拉普拉斯方程},即\(\dfrac{\partial^2 u}{\partial x^2} + \dfrac{\partial^2 u}{\partial y^2} = 0\),\(\dfrac{\partial^2 v}{\partial x^2} + \dfrac{\partial^2 v}{\partial y^2} = 0\);
    \end{itemize}
    (2)\(u(x,y) = C_1\),\(v(x,y) = C_2\)(\(C_1\),\(C_2\)为任意实数)是\(B\)上的两组\tboba{正交曲线簇}。
\end{theorem}

\begin{example}
    已知解析函数 \(f(z) = u(x,y) + \i v(x,y)\) 的虚部 \(v = \sqrt{-x + \sqrt{x^2 + y^2}}\),求其实部 \(u\) 和函数 \(f\)。
\begin{solution}
    改用极坐标 \(x = \rho\cos\varphi\),\(y = \rho\sin\varphi\),则有
    \[
        v = \sqrt{-\rho\cos\varphi + \sqrt{\rho^2\cos^2\varphi + \rho^2\sin^2\varphi}} = \sqrt{-\rho\cos\varphi + \rho} = \sqrt{2\rho} \sin \dfrac{\varphi}{2}
    \]
    则可求出 \( \dfrac{\partial v}{\partial \rho} = \dfrac{1}{\sqrt{2\rho}}\sin \dfrac{\varphi}{2} \),\(\dfrac{\partial v}{\partial \varphi} = \sqrt{\dfrac{\rho}{2}}\cos \dfrac{\varphi}{2}\)。因此有
    \[
        \dfrac{\partial u}{\partial \rho} = \dfrac{1}{\rho}\dfrac{\partial v}{\partial \varphi} = \dfrac{1}{\sqrt{2\rho}}\cos \dfrac{\varphi}{2},\qquad \dfrac{\partial u}{\partial \varphi} = -\rho\dfrac{\partial v}{\partial \rho} = -\sqrt{\dfrac{\rho}{2}}\sin \dfrac{\varphi}{2}
    \]
    得到
    \begin{align*}
        \dif u &= \dfrac{\partial u}{\partial \rho}\dif \rho + \dfrac{\partial u}{\partial \varphi}\dif \varphi = \dfrac{1}{\sqrt{2\rho}}\cos \dfrac{\varphi}{2}\dif \rho - \sqrt{\dfrac{\rho}{2}}\sin \dfrac{\varphi}{2}\dif \varphi \\
        &= \cos \dfrac{\varphi}{2}\dif \sqrt{2\rho} - \sqrt{2\rho} \dif \cos \dfrac{\varphi}{2} = \dif \left(\sqrt{2\rho}\cos \dfrac{\varphi}{2}\right)
    \end{align*}
    因此 
    \[
        u = \sqrt{2\rho}\cos \dfrac{\varphi}{2} + C
        = \sqrt{x + \sqrt{x^2 + y^2}} + C
    \]
    从而得到函数 \(f\) 为
    \begin{equation*}
        f(z) = \sqrt{x + \sqrt{x^2 + y^2}} + \i \sqrt{2\rho}\sin \dfrac{\varphi}{2} = \sqrt{x + \sqrt{x^2 + y^2}} + \i \sqrt{y + \sqrt{x^2 + y^2}} \qedhere
    \end{equation*}
\end{solution}
\end{example}

\Subsubsection{复变函数的多值性}

\de[多值函数]{
    若函数 \(w = f(z)\) 的定义域 \(E\) 中,对于某些 \(z\),存在多个值 \(w \) 与之对应,则称 \(w = f(z)\) 为\tboba{多值函数}。多值函数在每个宗量 \(z\) 处都有多个值,一般认为 \(f(z)\) 表示 \(z \) 处的全部值的集合。

    若定义在上面点集 \(E\) 上的单值函数 \(w_n = f_n(z) \in f(z)\)(\(\forall z \in E\)),则称 \(f_n(z)\) 为 \(f(z)\) 的一个\tboba{单值分支}。
}

\begin{example}
    设有函数 \(w = \sqrt{z}\),其中 \(|z| = \rho\),\(\arg z = \varphi\)。

    由 \(\sqrt{z} = \sqrt{\rho\e^{\i\mathop\mathrm{Arg} z}} = \sqrt{\rho}\e^{\i\mathop\mathrm{Arg} z/2} = \sqrt{\rho}\e^{\i(\varphi + 2k\pi)/2} = \sqrt{\rho}\e^{\i(\varphi/2 + k\pi)}\)(\(k \in \mathbb{Z}\)),
    知\(\sqrt{z}\) 有两个值:
    \[
        \begin{cases}[ll]
            k = 0\text{:} & w_1 = \sqrt{\rho}\e^{\i\varphi/2},\\
            k = 1\text{:} & w_2 = \sqrt{\rho}\e^{\i\left(\varphi/2 + \pi\right)} = -\sqrt{\rho}\e^{\i\varphi/2}
        \end{cases}
    \]
    因此,\(w = \sqrt{z}\) 有两个单值分支。

    设 \(z\) 从某一点 \(z_0\) 出发(对应地 \(w\) 从分支 \(w_1\) 上的 \(w_0 = \sqrt{|z_0|}\e^{\i\mathop\mathrm{arg} z_0/2}\) 出发),沿着闭合曲线 \(C\) 回到 \(z_0\)。
\end{example}

\subsection{复变函数积分}

\subsubsection{Cauchy积分定理}

\di[单连通区域 Cauchy 定理]{
    设函数 \(f(z)\) 在闭单连通区域 \(\overline{B}\) 上解析,
    则沿 \(\overline{B}\) 上任意分段光滑闭合曲线 \(\ell\),有 
    \begin{equation}
        \oint_\ell f(z)\dif z = 0
    \end{equation} 
}

\begin{theorem}
    设函数 \(f(z)\) 在单连通区域 \(B\) 上解析,且在 \(\overline{B}\) 上连续,则沿 \(\overline{B}\) 上任意分段光滑闭合曲线 \(\ell\),有 
    \begin{equation}
        \oint_\ell f(z)\dif z = 0
    \end{equation} 
\end{theorem}

\begin{example}
    计算积分 \(\displaystyle I = \oint_\ell (z-\alpha)^n \dif z\) (\(\alpha\) 为常数,\(n\) 为整数)。
    \label{eg: Cauchy积分公式的引理}
\begin{solution}
    (1)若回路 \(\ell\) 不包围点 \(\alpha\),则被积函数在 \(\ell\) 所围区域上解析,故积分 \(I = 0\)。

    (2)若回路 \(\ell\) 包围点 \(\alpha\),则当 \(n \ge 0\) 时,被积函数在 \(\ell\) 所围区域上解析,故积分 \(I = 0\);
    
    当 \(n \le -1\) 时,被积函数在 \(\ell\) 所围区域上有一个奇点 \(\alpha\)。由定义,存在以 \(\alpha\) 为中心、\(R\) 为半径的圆周 \(\gamma\) 在 \(\ell\) 所围区域内,因此在圆周 \(\gamma\) 上有 \(z - \alpha = R\e^{\i\varphi}\),以及
    \begin{equation*} 
        I = \oint_\ell (z-\alpha)^n \dif z = \oint_\gamma R^n\e^{\i n\varphi} \dif(\alpha + R\e^{\i\varphi}) = \i R^{n+1}\oint_0^{2\pi} \e^{\i n\varphi} \dif\varphi = \begin{cases}[ll]
            0, & n \neq -1,\\
            2\pi\i, & n = -1
        \end{cases}
        \qedhere
    \end{equation*}
\end{solution}
\end{example}

\subsubsection{Cauchy积分公式}

\di[单连通区域 Cauchy 积分公式]{
    设函数 \(f(z)\) 在闭单连通区域 \(\overline{B}\) 上解析,\(\ell\) 为 \(\overline{B}\) 的边界线,若点 \(\alpha\) 在 \(\ell\) 内部,则有
    \begin{equation}
        f(\alpha) = \dfrac{1}{2\pi\i} \oint_\ell \dfrac{f(z)}{z-\alpha} \dif z
    \end{equation}
}
\begin{proofs}
    由例~\ref{eg: Cauchy积分公式的引理}~可知,\(\displaystyle \oint_\ell \dfrac{\dif z}{z - \alpha} = 2 \pi \i\),因此有
    \[
        f(\alpha) = f(\alpha) \cdot \dfrac{1}{2\pi\i} \oint_\ell \dfrac{\dif z}{z - \alpha} = \dfrac{1}{2\pi\i} \oint_\ell \dfrac{f(\alpha)}{z-\alpha} \dif z
    \]
    于是只需证明 \(\displaystyle \dfrac{1}{2\pi\i} \oint_\ell \dfrac{f(z)-f(\alpha)}{z-\alpha} \dif z = 0\)。
    
    由于 \(\alpha\) 是区域 \(B\) 上一点,因此存在以 \(\alpha\) 为圆心、任意小 \(\varepsilon\) 为半径的小圆 \(C_\varepsilon \subset B\)。
    由 Cauchy 定理,有
    \[
        \left| \oint_\ell \dfrac{f(z)-f(\alpha)}{z-\alpha} \dif z \right| = \left| \oint_{C_\varepsilon} \dfrac{f(z)-f(\alpha)}{z-\alpha}  \dif z \right|
    \]
    我们有估计
    \[
        \left| \oint_{C_\varepsilon} \dfrac{f(z)-f(\alpha)}{z-\alpha}  \dif z \right| 
        \le \dfrac{\max\limits_{z \in C_\varepsilon} \left|f(z) - f(\alpha)\right|}{\varepsilon} 2\pi\varepsilon = 2\pi \max\limits_{z \in C_\varepsilon} |f(z) - f(\alpha)|
    \]

    令 \(\varepsilon \to 0\),即 \(C_\varepsilon \to \{\alpha\}\),由于 \(f(z)\) 连续,即 \(f(z) \to f(\alpha)\),\(\max\limits_{z \in C_\varepsilon} |f(z) - f(\alpha)| \to 0\),因此有
    \[
        \left| \oint_\ell \dfrac{f(z)-f(\alpha)}{z-\alpha} \dif z \right|
        = \lim_{\varepsilon \to 0} \left| \oint_{C_\varepsilon} \dfrac{f(z)-f(\alpha)}{z-\alpha}  \dif z \right| = 0
        \quad \Rightarrow \quad \oint_\ell \dfrac{f(z)-f(\alpha)}{z-\alpha} \dif z = 0
    \]
    故定理得证。
\end{proofs}

通常将 Cauchy 积分公式写为
\[
    f(z) = \dfrac{1}{2\pi\i} \oint_\ell \dfrac{f(\zeta)}{\zeta - z} \dif \zeta
\]

若 \(f(z)\) 在 \(\ell\) 所围区域上存在奇点,则需考虑挖去奇点后的复连通区域。在复连通区域上,\(f(z)\) 解析,显然 Cauchy积分公式仍然成立,只要将 \(\ell\) 理解为所有边界线并都取正向即可。

\begin{example}[2]
    考虑围线 \(C_R\colon |z|=R\),讨论不同 \(R\) 下的积分
    \[
        I = \oint_{C_R} \dfrac{1}{z^3(z+1)(z-1)} \dif z
    \]
\begin{solution}
    注意到被积函数在 \(z=0\),\(z=-1\),\(z=1\) 处都有奇点。
    
    (1)当 \(0<R<1\) 时,有
    \begin{align*}
        I_R &= \oint_{C_R} \dfrac{\dfrac{1}{(z+1)(z-1)}}{z^3} \dif z 
        = \dfrac{2\pi\i}{2!} \dfrac{\dif^2}{\dif z^2} \left(\dfrac{1}{(z+1)(z-1)}\right) \bigg|_{z=0} 
        = \dfrac{\pi\i}{2} \dfrac{\dif^2}{\dif z^2} \left(\dfrac{1}{z-1} - \dfrac{1}{z+1}\right) \bigg|_{z=0} \\
        &= \dfrac{\pi\i}{2} \left(\dfrac{2}{(z-1)^3} - \dfrac{2}{(z+1)^3}\right) \bigg|_{z=0} = -2\pi\i
    \end{align*}

    (2)当 \(R>1\) 时,在 \(C_R\) 内作三个小圆 \(C_{R_{-1}}\colon |z+1|=R_{-1}\),\(C_{R_0}\colon |z|=R_0\),\(C_{R_1}\colon |z-1|=R_1\),其中 \(R_{-1}+1<R\),\(R_1+1<R\),\(R_0+R_1<1\),\(R_0+R_{-1}<1\),
    则有
    \begin{align*}
        I_R &= \left(\ointctrclockwise_{C_{R_{-1}}} + \ointctrclockwise_{C_{R_0}} + \ointctrclockwise_{C_{R_1}}\right) \dfrac{1}{(z+1)(z-1)z^3} \dif z \\
        &= \ointctrclockwise_{C_{R_{-1}}} \dfrac{\dfrac{1}{z^3(z-1)}}{z+1} \dif z + \ointctrclockwise_{C_{R_0}} \dfrac{\dfrac{1}{(z+1)(z-1)}}{z^3} \dif z + \ointctrclockwise_{C_{R_1}} \dfrac{\dfrac{1}{z^3(z+1)}}{z-1} \dif z  \\
        &= 2\pi\i \left( \dfrac{1}{(-1)^3(-1-1)} - 1 + \dfrac{1}{1^(1+1)}\right) = 0
    \end{align*}

    (3)当 \(R=0\) 时,积分不存在。
\end{solution}
\end{example}

\begin{example}[1]
    \textbf{代数学基本定理\quad} 证明在 \(z\) 平面上的 \(n\) 次多项式 \(p_n(z) = a_n z^n + a_{n-1} z^{n-1} + \cdots + a_1 z + a_0\) (\(a_n \neq 0\))至少有一个零点。
    \begin{proofs}
        用反证法。若 \(p_n(z)\) 无零点,则有 \(\dfrac{1}{p_n(z)}\) 在整个复平面上解析。由于
        \[
            \lim_{|z| \to \infty} p_n(z) = \lim_{|z| \to \infty} z^n \left(a_n + \dfrac{a_{n-1}}{z} + \cdots + \dfrac{a_1}{z^{n-1}} + \dfrac{a_0}{z^n}\right) = \infty
        \]
        因此
        \[
            \lim_{|z| \to \infty} \dfrac{1}{p_n(z)} = 0
        \]
        由 Liouville 定理,\(\dfrac{1}{p_n(z)}\) 为常数,即 \(p_n(z)\) 为常数,与 \(a_n \neq 0\) 矛盾。因此,\(p_n(z)\) 至少有一个零点。
    \end{proofs}
\end{example}

\subsection{Taylor级数和Laurent级数}

\subsubsection{复数项级数}

\de[复数项级数]{
    设有复数项级数
    \[
        \sum_{n=0}^\infty z_n = z_0 + z_1 + z_2 + \cdots + z_n + \cdots
    \]
    其中 \(z_n = a_n + \i b_n\),\(a_n\) 和 \(b_n\) 为实数。若级数通项的实部\(a_n\) 和虚部 \(b_n\) 的级数分别收敛,则称级数 \(\sum\limits_{n=0}^\infty z_n\) \,\tboba{收敛},并定义其和为
    \[
        \sum_{n=0}^\infty z_n = \sum_{n=0}^\infty a_n + \i \sum_{n=0}^\infty b_n
    \]
}

\begin{theorem}
    \textbf{\textup{Cauchy 收敛准则\quad}} 级数 \(\sum\limits_{n=0}^\infty z_n\) 收敛,当且仅当对任意 \(\varepsilon > 0\),存在正整数 \(N\),使得当 \(n > N\) 时,对任意 \(p \in \mathbb{N}^*\),有
    \[
        \left| \sum_{n+1}^{n+p} z_n \right| < \varepsilon
    \]
\end{theorem}

\begin{definition}
    如果级数 \(\sum\limits_{n=0}^\infty z_n\) 收敛,且级数 \(\sum\limits_{n=0}^\infty |z_n|\) 也收敛,则称级数 \(\sum\limits_{n=0}^\infty z_n\) \,\tboba{绝对收敛}。
\end{definition}

绝对收敛的级数必然收敛,且求和的先后次序可以任意改变。

\begin{theorem}
    设有两个级数 \(\sum\limits_{n=0}^\infty z_n\) 和 \(\sum\limits_{n=0}^\infty w_n\),若级数 \(\sum\limits_{n=0}^\infty z_n\) 和 \(\sum\limits_{n=0}^\infty w_n\) 分别绝对收敛于 \(A\)、\(B\),则其逐项相乘的级数 \(\sum\limits_{n=0}^\infty \sum\limits_{k=0}^\infty z_n w_k\) 绝对收敛于 \(AB\)。
\end{theorem}

\de[函数项级数]{
    设有函数项级数
    \[
        \sum_{n=0}^\infty w_n(z) = w_0(z) + w_1(z) + w_2(z) + \cdots + w_n(z) + \cdots
    \]
    若对某个区域 \(B\) 或某根曲线 \(\ell\) 上的每一点 \(z\),级数 \(\sum\limits_{n=0}^\infty w_n(z)\) 收敛,则称级数 \(\sum\limits_{n=0}^\infty w_n(z)\) 在区域 \(B\) 或沿曲线 \(\ell\) \,\tboba{收敛}。
}

\begin{theorem}
    \textbf{\textup{Cauchy 收敛准则}\quad} 函数项级数 \(\sum\limits_{n=0}^\infty w_n(z)\) 在区域 \(B\) 或沿曲线 \(\ell\) 上收敛,当且仅当对任意 \(\varepsilon > 0\),存在 \(N(z)\) 使得当 \(n > N(z)\) 时,对任意 \(p \in \mathbb{N}^*\),有
    \[
        \left| \sum_{n+1}^{n+p} w_n(z) \right| < \varepsilon
    \]
    若选取的 \(N(z)\) 与点 \(z\) 无关,则称级数 \(\sum\limits_{n=0}^\infty w_n(z)\) 在区域 \(B\) 或沿曲线 \(\ell\) 上\,\tboba{一致收敛}。
\end{theorem}

\begin{theorem}
    (1)若在区域 \(B\) 或曲线 \(\ell\) 上 一致收敛的级数 \(\sum\limits_{n=0}^\infty w_n(z)\) 每一项都是 \(B\) 或 \(\ell\) 上的连续函数,则级数的和函数 \(w(z) = \sum\limits_{n=0}^\infty w_n(z)\) 也是 \(B\) 或 \(\ell\) 上的连续函数。

    (2)若在曲线 \(\ell\) 上 一致收敛的级数 \(\sum\limits_{n=0}^\infty w_n(z)\) 每一项都是 \(\ell\) 上的连续函数,则级数可以沿 \(\ell\) 逐项积分,即
    \[
        \int_\ell \sum_{n=0}^\infty w_n(\zeta) \dif \zeta = \sum_{n=0}^\infty \int_\ell w_n(\zeta) \dif \zeta
    \]

    (3)若在闭区域 \(\overline{B}\) 上一致收敛的级数 \(\sum\limits_{n=0}^\infty w_n(z)\) 每一项都是 \(\overline{B}\) 上的解析函数,则级数的和函数 \(w(z) = \sum\limits_{n=0}^\infty w_n(z)\) 也是 \(\overline{B}\) 上的解析函数,且级数可以逐项求导,即
    \[
        \dfrac{\dif^n}{\dif z^n} \sum_{n=0}^\infty w_n(z) = \sum_{n=0}^\infty \dfrac{\dif^n}{\dif z^n} w_n(z)
    \]
    其中各阶导数 \(\dfrac{\dif^n}{\dif z^n} w_n(z)\) 也在 \(\overline{B}\) 上一致收敛。
\end{theorem}

\begin{definition}
    若函数项级数 \(\sum\limits_{n=0}^\infty w_n(z)\) 在区域 \(B\) 或沿曲线 \(\ell\) 上所有点 \(z\) 处都满足通项 \(|w_n(z)| \le M_n\),而正常数项级数 \(\sum\limits_{n=0}^\infty M_n\) 收敛,则称级数 \(\sum\limits_{n=0}^\infty w_n(z)\) 在区域 \(B\) 或沿曲线 \(\ell\) 上\tboba{绝对且一致收敛}。
\end{definition}

\de[幂级数]{
    设有函数项级数
    \[
        \sum_{k=0}^\infty a_k(z-z_0)^k = a_0 + a_1(z-z_0) + a_2(z-z_0)^2 + \cdots + a_k(z-z_0)^k + \cdots
    \]
    其中 \(a_k\) 为常数,\(z_0\) 为常数,这样的级数称为以 \(z_0\) 为中心的\tboba{幂级数}。
}

\begin{theorem}
    \textbf{\textup{幂级数的绝对收敛条件}\quad} 对幂级数 \(\sum\limits_{k=0}^\infty a_k(z-z_0)^k\),若存在正数 \(R = \lim\limits_{k \to \infty} \dfrac{|a_k|}{|a_{k+1}|}\),则 \(|z-z_0|<R\) 时 \(\sum\limits_{k=0}^\infty a_k(z-z_0)^k\) 绝对收敛,而 \(|z-z_0|>R\) 时级数发散。
\end{theorem}

\zhu[收敛半径]{
    幂级数 \(\sum\limits_{k=0}^\infty a_k(z-z_0)^k\) 的\tboqi{收敛半径}\,\(R\) 可定义为
    \[
        R = \lim_{k \to \infty} \left|\dfrac{a_k}{a_{k+1}}\right| 
        \qquad \text{或} \qquad
        R = \limsup\limits_{k \to \infty} \dfrac{1}{\sqrt[k]{|a_k|}}
    \]
    圆 \(|z-z_0|=R\) 称为幂级数的\tboqi{收敛圆}。
}

幂级数在收敛圆的内部绝对且一致收敛,进而在收敛圆内单值解析,于是可以逐项积分、逐项微分,且逐项积分、微分不改变收敛半径。

\begin{example}[3]
    讨论幂级数 \(\sum\limits_{k=0}^\infty z^k \cos \i k\) 的收敛半径。
\begin{solution}
    幂级数 \(\sum\limits_{k=0}^\infty z^k \cos \i k\) 的系数为 \(a_k = \cos \i k = \dfrac{\e^{k} + \e^{-k}}{2}\),因此收敛半径为
    \[
        R = \lim_{k \to \infty} \dfrac{1}{\sqrt[k]{|a_k|}} = \lim_{k \to \infty} \dfrac{\sqrt[k]{2}}{\sqrt[k]{\e^k + \e^{-k}}}
    \]
    注意到 \(k \to \infty\) 时,\(\e < \sqrt[k]{\e^k + \e^{-k}} < \sqrt[k]{2\e^k} = \sqrt[k]{2}\e \to \e\),因此收敛半径为 \(R = \dfrac{1}{\e}\)。
\end{solution}
\end{example}

\subsubsection{Taylor级数}

\di[解析函数的Taylor展开]{
    设函数 \(f(z)\) 在点 \(z_0\) 为圆心的圆盘 \(C_R \colon |z-z_0|<R\) 上解析,那么在圆盘 \(C_R\) 内任意点 \(z\),函数 \(f(z)\) 可展开为幂级数
    \[
        f(z) = \sum_{n=0}^\infty a_n (z-z_0)^n
    \]
    其中
    \begin{equation}
        a_n = \dfrac{1}{2\pi\i} \oint_{C_{R_1}} \dfrac{f(\zeta)}{(\zeta-z_0)^{n+1}} \dif \zeta = \dfrac{1}{n!} f^{(n)}(z_0)
    \end{equation}
    \(C_{R_0}\) 为以 \(z_0\) 为圆心、包含点 \(z\) 的圆周。
}

\begin{proofs}
    根据 Cauchy 积分公式,有 \(\displaystyle f(z) = \dfrac{1}{2\pi\i} \oint_{C_{R_1}} \dfrac{f(\zeta)}{\zeta-z} \dif \zeta\),可先将 \(\dfrac{1}{\zeta-z}\) 展开为幂级数,即
    \[
        \dfrac{1}{\zeta-z} = \dfrac{1}{(\zeta-z_0) - (z-z_0)} = \dfrac{1}{\zeta-z_0} \dfrac{1}{1-\dfrac{z-z_0}{\zeta-z_0}} = \dfrac{1}{\zeta-z_0} \sum_{n=0}^\infty \left(\dfrac{z-z_0}{\zeta-z_0}\right)^n = \sum_{n=0}^\infty \dfrac{(z-z_0)^n}{(\zeta-z_0)^{n+1}}
    \]
    这里由于 \(\zeta\) 在以 \(z_0\) 为圆心、包含点 \(z\) 的圆周 \(C_{R_0}\) 上,容易知道满足 \(\left|\dfrac{z-z_0}{\zeta-z_0}\right| < 1\)。代入 Cauchy 积分公式,得到
    \begin{align*}
        f(z) &= \dfrac{1}{2\pi\i} \oint_{C_{R_1}} f(\zeta) \sum_{n=0}^\infty \dfrac{(z-z_0)^n}{(\zeta-z_0)^{n+1}} \dif \zeta 
        = \sum_{n=0}^\infty (z-z_0)^n \dfrac{1}{2\pi\i} \oint_{C_{R_1}} \dfrac{f(\zeta)}{(\zeta-z_0)^{n+1}} \dif \zeta \\
        &= \sum_{n=0}^\infty \dfrac{f^{(n)(z_0)}}{n!} (z-z_0)^n
    \end{align*}
    从而可证。
\end{proofs}

\begin{example}
    考虑函数 \(f(z) = \ln z\) 在点 \(z_0 = 1\) 附近的 Taylor 展开。
\begin{solution}
    多值函数 \(\ln z\) 的支点为 \(z = 0,\infty\),因此在 \(z_0 = 1\) 附近的 Taylor 展开的圆盘应该避开这两个支点。取 \(R = 1\),则在圆盘 \(C_1\) 内有
    \begin{align*}
        &f(z) = \ln z && f(1) = \ln 1 = 2\pi n\i \\
        &f^{(1)}(z) = \dfrac{1}{z} && f^{(1)}(1) = 1 \\
        &f^{(2)}(z) = -\dfrac{1}{z^2} && f^{(2)}(1) = -1 \\
        &\vdots && \vdots \\
        &f^{(k)}(z) = (-1)^{k-1} \dfrac{(k-1)!}{z^k} && f^{(k)}(1) = (-1)^{k-1} (k-1)!
    \end{align*}
    因此在圆盘 \(C_1\) 内有
    \[
        \ln z = \sum_{n=0}^\infty (-1)^{n-1} \dfrac{(n-1)!}{n!} (z-1)^n = \sum_{n=0}^\infty (-1)^{n-1} \dfrac{(z-1)^n}{n}
    \]
\end{solution}
\end{example}



\newpage
%----------------------------------------------------------
\section{积分变换}


\newpage
%----------------------------------------------------------
\section{数理方程}


\newpage
%----------------------------------------------------------
\section{特殊函数}


%----------------------------------------------------------
\end{document}