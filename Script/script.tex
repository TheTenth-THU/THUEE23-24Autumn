% 本模板采自 http://kevindonnelly.org.uk/resources/playscript.tex

\documentclass[10pt, a5paper, oneside]{memoir}
    % 使用 memoir 文档类,并定义字号为 10pt,纸张大小为 A5,单面打印。
    % http://www.ctan.org/tex-archive/macros/latex/contrib/memoir
    % 许多文学作品的排版都可以使用 memoir 文档类,如小说、剧本、诗歌等。本模板来自 Peter Wilson 的《Some Examples of Title Pages》(http://www.ctan.org/tex-archive/info/latex-samples/TitlePages)。

\usepackage[english]{babel}
    % 用于支持多种语言的宏包。
    % http://www.ctan.org/tex-archive/macros/latex/required/babel
\usepackage{enumitem}
    % 使用 enumitem 宏包,可以自定义列表的样式。
    % http://www.ctan.org/tex-archive/macros/latex/contrib/enumitem
\usepackage[fontset=none]{ctex}
    % 使用 ctex 宏包,可以在 LaTeX 中使用中文。
    % http://www.ctan.org/tex-archive/macros/latex/contrib/ctex
\usepackage{amsmath}
    % 使用 amsmath 宏包,可以使用更多数学公式。
    % http://www.ctan.org/tex-archive/macros/latex/required/amslatex/math
\usepackage[
    top=10mm, bottom=10mm, left=10mm, right=10mm, 
    headheight=5mm,
    includehead, includefoot
]{geometry}
    % 设置版式。
    % http://www.ctan.org/tex-archive/macros/latex/contrib/geometry
\usepackage{multirow}
    % 使用 multirow 宏包,可以在表格中合并单元格。
    % http://www.ctan.org/tex-archive/macros/latex/required/tools

\setmonofont{Iosevka}
\setCJKmainfont{FZXSSK.TTF}[BoldFont={SourceHanSerifCN-Bold.otf}, ItalicFont={FZXKTK.TTF}, BoldItalicFont={汉仪颜楷W.ttf}]
\setCJKsansfont{汉仪文黑-45W.ttf}[BoldFont={汉仪文黑-75W.ttf}, ItalicFont={FZYanZQKSJF.TTF}]
\setCJKmonofont{LXGWNeoXiHei.ttf}
\newCJKfontfamily{\kai}{FZXKTK.TTF}[BoldFont={汉仪颜楷W.ttf}, ItalicFont={方正清刻本悦宋 简繁.TTF}, BoldItalicFont={FZYanZQKSJF.TTF}]

\newlength{\drop}
    % 定义 \drop 长度变量,用于标题页的垂直位置调整。
    % 若不希望使用 \drop,可参考 http://wiki.lyx.org/LyX/UsingMemoirInLyX

\chapterstyle{demo2}
    % 查看 memoir 手册第 92 页以了解更多章节样式。
\setlength{\beforechapskip}{\baselineskip}

\pagestyle{myheadings}

\setlength{\parindent}{0pt}

\renewcommand{\printtoctitle}[1]{\centering\Large\bfseries 目录}  
    % 设置目录页的标题。
% \renewcommand{\printtoctitle}[1]  
    % 删去目录页的标题。

\pagenumbering{gobble} 
    % 不显示页码,直到另有说明。


\setitemize[1]{
    itemsep=0pt, topsep=0em, partopsep=0pt, parsep=0pt, 
}




% 设置标题页的样式。
\newcommand*{\titleGM}{
    \begingroup
        % 使用 Gentle Madness 标题页样式。
    \thispagestyle{plain}
    \drop = 0.1\textheight
    % \vspace*{\baselineskip}
    % \vfill
    \hbox{%
        \hspace*{0.2\textwidth}%
        \rule{1pt}{\textheight}
        \hspace*{0.05\textwidth}%
        \parbox[b]{0.75\textwidth}{
            \vbox{%
                % 剧本的主标题
                \vspace{\drop}{
                    \noindent\HUGE\bfseries 频率之外{\Large(暂名)}
                }\\
                % \vspace{\drop}{
                %     \noindent\HUGE\bfseries Title of the play \\[0.5\baselineskip] 
                %     over two lines
                % }\\
                % 
                % 剧本的副标题
                [1\baselineskip]{\huge\scshape\slshape Beyond the Frequency}\\
                % [2\baselineskip]{\huge\itshape Subtitle of the play \\[0.5\baselineskip] over two lines}\\
                % 
                % 剧本的类型/作者
                [2\baselineskip]{\Large 科幻(且希望是但还不是)喜剧}
                \par\vspace{0.5\textheight}
                % [4\baselineskip]{
                %     \Large First Author \\[0.5\baselineskip] 
                %     Second Author \\[0.5\baselineskip] 
                %     Third Author \\
                % }\par\vspace{0.5\textheight}
                % 
                % 剧本的出版信息
                {\noindent \textbf{无38} \\[0.5\baselineskip] 
                \textbf{\today}}\\[\baselineskip]
            }% end of vbox
        }% end of parbox
    }% end of hbox
    % \vfill
    \null
    \endgroup
}

\hbadness=10000
    % 忽略盒子溢出的警告。

\newcommand{\stage}[1]{%
    % 舞台说明。
    \\[0.4\baselineskip] \textit{#1}%
}
\newcommand{\dialog}[2][旁白]{%
    % 对话。
    \item[#1] #2%
}
\newcommand*{\direction}[1]{\textit{(#1)}}
    % 动作说明。

\newcounter{scene}
\setcounter{scene}{0}
    % 场景计数器,从 1 开始。
\newcommand{\scene}[2][]{
    % 场景。
    \stepcounter{scene}
        % 场景计数器加 1。
    \section*{第 \thescene 场\; #1}
        % 场景标题。
    % \section*{\hfill\textit{SCENE 1}}
        % 场景标题右对齐。
    \addcontentsline{toc}{section}{SCENE \thescene\quad #1}
        % \addcontentsline 命令将场景添加到目录中。
    \begin{description}[align=left, itemsep=1ex, leftmargin=2em, itemindent=-1em, labelsep=1em]
        % 设置一个描述列表来容纳场景的对话。增加列表项之间的间距,并将左边距设置为 1cm。
        \item[] \hfill 
        #2
    \end{description}
        % 结束描述列表的对话。
    \vskip 1cm
        % 场景标题和对话之间的空白。
}

\newcounter{act}
\setcounter{act}{0}
    % 幕计数器,从 1 开始。
\newcommand{\act}[2][]{
    \stepcounter{act}
        % 幕计数器加 1。
    %%%%%%%%%%%%%%%%%%%%%%%%%%%%%%%%%
    \chapter*{第 \theact 幕\; #1}
    %%%%%%%%%%%%%%%%%%%%%%%%%%%%%%%%%
    \addcontentsline{toc}{chapter}{ACT \theact\quad #1}
    \setcounter{scene}{0}
        % 重置场景计数器。
    #2
        % 打印出幕的内容。
}


\begin{document}

%-----------------------------------------------------------

\titleGM
    % 标题页。

%-----------------------------------------------------------

\setlength{\parskip}{0.5\baselineskip}
    % 设置段间距为 1/2 行高。

\markright{\textsc{Beyond the Frequency}}
    % 生成每一页的页眉。

\tableofcontents*
    % 目录。

\pagenumbering{roman}
    % 使用罗马数字编号页码。

\clearpage

%%%%%%%%%%%%%%
\chapter*{角色}
%%%%%%%%%%%%%%%
    % 角色表。
\addcontentsline{toc}{chapter}{角色}

\textbf{A组},主人公(姓名待定),从Q中学毕业后进入T大学电子系学习,曾因T大学科技营与嵌入式开发结缘,但进入大学后逐渐感到不知所措,变得迷茫与颓废。
\begin{itemize}
    \item \textbf{AC},时间点\(C\)的(姓名待定),T大学电子系学生。
    \item \textbf{AP},时间点\(P\)的(姓名待定),Q中学学生。
    \item \textbf{AF},时间点\(F\)的(姓名待定),学术上「小有成就」。
\end{itemize}

\begin{table}[!ht]
    \begin{tabular}{p{0.15\textwidth}<{\centering}|p{0.2\textwidth}<{\centering}|p{0.55\textwidth}<{\centering}}
        \toprule
        ~ & & \\[-0.7em]
            % 这一行增加行高
        \textbf{A组} & \textbf{A} & \textit{肖怡君} \\[0.3em]
        \bottomrule
    \end{tabular}
\end{table}

\textbf{B组},AC 的同学和队友,与 AC 同队参加硬件设计与科技营活动。

\begin{table}[!ht]
    \begin{tabular}{p{0.15\textwidth}<{\centering}|p{0.2\textwidth}<{\centering}|p{0.55\textwidth}<{\centering}}
        \toprule
        ~ & & \\[-0.7em]
            % 这一行增加行高
        \multirow{2}*{\textbf{B组}} & \textbf{B0} & \textit{李硕实} \\[0.3em]
        ~                           & \textbf{B1} & \textit{陈禛兴} \\[0.3em]
        \bottomrule
    \end{tabular}
\end{table}

\textbf{C组},科技营带班同学,为 AP 带去嵌入式开发的启蒙。其中,C0 负责指导 AP 所在的小组,并为 AP 提供了精神上的支持。

\begin{table}[!ht]
    \begin{tabular}{p{0.15\textwidth}<{\centering}|p{0.2\textwidth}<{\centering}|p{0.55\textwidth}<{\centering}}
        \toprule
        ~ & & \\[-0.7em]
            % 这一行增加行高
        \multirow{2}*{\textbf{C组}} & \textbf{C0} & \textit{刘晋昊} \\[0.3em]
        ~                           & \textbf{C1~C4} & \textit{翟仲浩 潘 浩 陶 硕 夏子凯} \\[0.3em]
        \bottomrule
    \end{tabular}
\end{table}

\textbf{D组},AP 的同学,与 AP 同组参加 C 组角色的科技营。在 C 组到来开展教学的前夜,D0~D2 与 AP 一起在教室外好奇地围观。

\begin{table}[!ht]
    \begin{tabular}{p{0.15\textwidth}<{\centering}|p{0.2\textwidth}<{\centering}|p{0.55\textwidth}<{\centering}}
        \toprule
        ~ & & \\[-0.7em]
            % 这一行增加行高
        \multirow{2}*{\textbf{D组}} & \textbf{D0~D2} & \textit{白 澈 王禹博 杨梓琨} \\[0.3em]
        ~ & \textbf{D3、D4} & \textit{单 硕 郝亚鹏} \\[0.3em]
        \bottomrule
    \end{tabular}
\end{table}

\textbf{E组},AC 科技营授课学生。在 AP 开展教学的前夜,E 组的同学们在教室外好奇地围观,E0 像 AP 那样被推出。

\begin{table}[!ht]
    \begin{tabular}{p{0.15\textwidth}<{\centering}|p{0.2\textwidth}<{\centering}|p{0.55\textwidth}<{\centering}}
        \toprule
        ~ & & \\[-0.7em]
            % 这一行增加行高
        \multirow{2}*{\textbf{E组}} & \textbf{E0} & \textit{任秋莼} \\[0.3em]
        ~                           & \textbf{E1~E4} & \textit{姜 为 黄浩然 邰伟宸 吴耀宇} \\[0.3em]
        \bottomrule
    \end{tabular}
\end{table}


%%%%%%%%%%%%%%%%%%%%%
\chapter*{第一稿说明}
%%%%%%%%%%%%%%%%%%%%%
\addcontentsline{toc}{chapter}{前言}
\addcontentsline{toc}{section}{第一稿说明}

首先必须抱歉地说明,由于时间仓促,这一稿只能大致勾勒出剧情的轮廓,剧本的各种细节还有待进一步完善。

在这个剧本中,我试图通过一个经典的科幻情节,用两个回环讲述一个关于成长、关于教育的故事。主人公进入大学后,被成绩、表格等各种琐事压垮,逐渐迷失了自己的热爱,也忘记了当初为何会来到这所大学。未来的他借用过去的他,帮现在的他找回当初的热情,这是其一;过去的支教队帮助过去的他走出第一步,现在的他乃至未来的他也要去支教队帮助更多的学生,这是其二。这两个回环交织在一起,构成了这个故事的主线。

剧本所基于的背景是硬设大赛和与之相关的单片机支教实践,这是我在大学时期参与过的两个活动,也是我认为对于电子系大学生而言非常有意义的两个活动。但是强调这些,或许会让剧本显得有些「官方」,而我并不希望把剧本拍成「宣传片」。为此,在人物日常的对白和互动中,还需要更多的生活化、幽默化的细节。在后续的修改中,这些部分会交由演员自己创作修改,以求呈现得更加自然、生动、接地气。

{\hfill\kai 2024年9月27日}

% The preface to a play seems generally to be considered as a kind of closet-prologue in which the author solicits that indulgence from the reader which he had before experienced from the audience.

% I need scarcely add that the circumstance alluded to was the withdrawing of the piece to remove those imprefections which were too obvious to escape reprehension.

% With regard to some particular passages which seemed generally disliked, I confess that if I felt any emotion of surprise at the disapprobation, it was not that they were disapproved of, but that I had not before perceived that they deserved it.

\clearpage

%%%%%%%%%%%%%%%%%%%%%
\chapter*{第二稿说明}
%%%%%%%%%%%%%%%%%%%%%
\addcontentsline{toc}{section}{第二稿说明}

在第一稿的基础上,我对剧本进行了一些修改。

首先,受限于拍摄场地的条件,经过取舍,暂定还是直白地把剧本的背景设定在大学校园,而 A 的回忆则设定为在大学科技冬令营中。
这样一来,剧本的背景就更加贴近大学生的生活,也更容易找到合适的拍摄场地。



\clearpage
%-----------------------------------------------------------
\pagenumbering{arabic}
    % 开始使用阿拉伯数字编号页码。

%%%%%%%%%%%%%%%
\chapter*{概要}
%%%%%%%%%%%%%%%
    % Print out a page with a summary of the plot.
\addcontentsline{toc}{chapter}{概要}

儿时(时间点\(P\)),A 参与 T 大学科技冬令营,在 T 大学同学的带领下尝试进行硬件设计时,将一首诗编码发送到太空中。几年后(时间点\(C\)),一段相似的回波又被考入该大学电子系、正调试着类似硬设作品的 A 捕获。与此前不同的是,现在的 A 没有了当年(\(P\))的探索热情,已经被数据、表格、成绩压垮。也因此,这成为了 A 生活轨迹的一个转折点。多年之后(时间点\(F\))的 A 完成了时空的沟通,在信息的茫茫串流中找到了当年(\(P\))编码的那一首诗。他回想起当初(\(C\))的迷茫与颓丧,把这首诗送回给那时(\(C\))的自己。

\textbf{第 1 幕}:重逢。时间点\(C\),A 在实验室天台与同学一起监听天线接收到的信号,突然间信号变得异常清晰,他意识到这是当年编码的那首诗。

\textbf{第 2 幕}:缘起。时间点\(P\),A 在科技营学长的带领下尝试进行硬件设计,将一首诗编码发送到太空中。

\textbf{第 3 幕}:复燃。时间点\(C\),重新振作起来的 A 开始调试硬设作品,并随支教队将启蒙知识传递给后来的学生。

\textbf{第 4 幕}:再会。时间点\(F\),A 完成了时空的沟通,找到了当年编码的那首诗,回想起当初的迷茫与颓丧,把这首诗送回给当时的自己。

%%%%%%%%%%%
\act[重逢]{
%%%%%%%%%%%
    \scene{
        \stage{
            \textup{
                \textbf{时间点\(\boldsymbol{C}\)},T大学电子系实验室天台。
            }
            夜幕降临,繁星点点,AC 和几位同学正围坐在一台复杂的天线设备旁,调试着设备。
        }
        \dialog[AC]{
            \direction{不耐烦地} 要不就这样,差不多得了吧。 \direction{把说明书丢到一边}
        }
        \dialog[B0]{
            再看看呢,没多少地方没排查过了。\direction{拿起说明书}
        }
        \dialog[B1]{
            大不了再把排线检查一遍嘛。
        }
        \dialog[AC]{
            \direction{一脸厌倦、无奈} 何必呢。再得不到结果,也得先准备答辩去了。
        }
        \dialog[B0]{
            但是连一个像样的信号都没收到,答辩咱能说啥呢?
        }
        \dialog[B1]{
            给评委们谈谈背景,谈谈理想,谈谈愿景嘛。\direction{装模作样起来}咳咳,「我们的设备虽然没收到可识读的信号,但这只是暂时的技术性调整,我们的项目仍然前景远大、前途光明,背后是对未知的探索、对未来的憧憬,承载了我们最初……」
        }
        \dialog[AC]{
            \direction{打断} 你打住!\direction{笑} 说起来,咱们这个项目最初是要干啥来着?
        }
        \dialog[B0]{
            最初你是说要做一个通信终端,然后几百块的报销额度只能做一端,所以就只做了一个接收端。
        }
        \dialog[B1]{
            然后只有接收端就没有办法控制测试信号,所以就只有在天台上看微波背景辐射了。
        }
        \dialog[AC]{
            但是就我们手上这些东西 \direction{把黑盒盖子打开},板子是串口不够刚换的,线头是断了又重新焊的,淘宝买的芯片问客服还说测试程序是外包的,又怎么捕获到背景辐射那微小的波动呢?
            \direction{把盖子推回但没有合上,激动而无力}
        }
        \dialog[B0]{
            也确实是,你看 \direction{指着说明书} 这东西,说是说明书,其实就是 datasheet,
            错误码没解释,示例程序没输出,光给些电压、电流、温度的,
            除了我们用不到的各种 data 就只有 sh……
            \direction{被捂嘴} 但既然我们都到这一步了,还是再试试呗。
        }
        \dialog[B1]{
            从另一个角度想,需要我们去解决的问题不也没多少嘛。\direction{推开盖子,开始检查} 反正也就这些接线、程序、电源、天线是我们能控制的。看到一点比现在的噪声大的波动,我们就算成功了。
        }
        \stage{
            B1 一个接一个地检查着线头,B0 见状也翻看起程序,AC 颓丧地坐在一旁,看着他们,又看向远方。
        }
        \stage{
            重新上传几次程序之后,当 B1 将某个公头拔出又重新插上时,信号突然变得异常清晰。随后,串口绘图仪有规律地跳动起来。
        }
        \dialog[AC]{
            \direction{随意地低头,看到串口绘图仪的图像} 欸这是……
        }
        \dialog[B1]{
            \direction{看着绘图仪} 刚才我们讨论的预期是……这样吗?
        }
        \stage{
            脑中突然有了某个想法的 AC 起身抱起电脑,快速调整起程序。再次上传程序后,串口监视器上 0 和 1 滚动起来。
        }
        \dialog[AC]{
            \direction{意外,出神} 果然啊……
        }
    }
}

%%%%%%%%%%%
\act[缘起]{
%%%%%%%%%%%
    \scene{
        \stage{
            \textup{
                \textbf{时间点\(\boldsymbol{P}\)},Q中学机房。
            }
            渐渐入夜,支教实践支队的几位同学正在一台一台地手动安装软件,为即将开始的支教活动做准备。机房窗外,刚下晚自习的 AP 带着D组同学们背着包、抱着书,靠在窗边向里张望。
        }
        \dialog[C1]{
            \direction{在不同进度的三台电脑前来回操作} 所以,这儿的终端管理软件是真的不能直接把软件包发下去吗?
        }
        \dialog[C2]{
            \direction{在一台电脑前操作} 那样更慢。这里主机和从机本来连接就不大好,你看那边还有几台点一键开机没反应的。
        }
        \dialog[C3]{
            \direction{在另一台电脑前操作} 你们下好之后,记得拿个示例程序编译上传一下,教案里随便一个就行,只需要看看软件包有没有问题。\direction{看向窗外,发现了 AP和同学们} 外面的同学们好像很好奇的样子呢。
        }
        \stage{
            窗外的同学们看到被发现后,有些尴尬地哄笑了一下,相视一笑,把带他们来的 AP 推到了门前,哄笑着跑开到远处。
            队员们看着 AP 的背影,笑了一下,C0 前去拉住了 AP,带他进了机房。     
        }
        \dialog[C0]{
            怎么,是好奇我们在干嘛吗?
        }
        \dialog[AP]{
            \direction{尴尬} 嗯,我……我只是路过,看到你们在这儿,就想看看。
        }
        \dialog[C0]{
            \direction{回到电脑前,插好 UNO 板,把自己的测试程序上传} 具体在做什么,等到后面的课程你就知道了。相信我们啦,几天之后,你会掌握一些还算有点用处的技能的。
        }
        \stage{
            AP 用力地点了点头,看着屏幕上跳动的图像和滚动的数字。
        }
    }
    \scene{
        \stage{
            \textup{
                \textbf{时间点\(\boldsymbol{P}\)},Q中学机房。
            }
            自主设计阶段,AP 和同组的同学正在兴奋地讨论着他们的设计方案。这一组的指导助教 C0 在一旁默默地听着,同时在电脑上调试着自己的程序。
        }
        \dialog[D0]{
            \direction{摆弄着几个舵机} 
            拿这几个舵机做个按摩椅模型,怎么样?
        }
        \dialog[D1]{
            不大行吧,刚刚老师不是说这个舵机工作一段时间就会抽风嘛。而且,只做一个模型,也没有实际功能,不太好哦。
        }
        \dialog[D2]{
            那做个闹钟呢?把电机放在闹钟上,早上闹钟响了,电机就把你的被子给扯掉,这样就不会再睡过头了。
        }
        \dialog[D3]{
            \direction{众笑}
            这小电机哪有那么大劲啊。
            要不做个收纳箱吧,把舵机放在箱子里,然后用板子控制箱子的开合,用摇杆上上下下左左右右AABB当密码,这样就有点意思了,也有用处。
        }
        \dialog[D4]{
            做个音乐盒呢,我看老师之前演示的蜂鸣器编曲还挺方便的。
        }
        \dialog[AP]{
            要不做点大的,比如,做个通信终端?我想看看书上那种把电容打开、把电感拉直的天线到底是怎么用的。
        }
        \dialog[D1]{
            可我们硬件就这么多了,应该很难攒出两台来吧。
        }
        \dialog[D2]{
            而且,UNO 板也就这么大功率,通信信号估计不会太强哦。
        }
        \dialog[D3]{
            咱们又没有专业的通信天线,噪声估计能把信号给淹没了都。通用的那些通信协议也不是那么容易实现的。
        }
        \dialog[AP]{
            \direction{思考} 哎呀,关键还是得要有意思嘛。这些问题 C0 老师肯定能帮我们解决的。\direction{笃定,看向 C0}
        }
        \dialog[C0]{
            \direction{突然听到自己的名字} 嗯?我咋啦?
        }
        \dialog[D0]{
            没咋没咋。\direction{看到 C0 的电脑屏幕} 老师你在调什么程序呢?
        }
        \dialog[C0]{
            \direction{指着屏幕} 挺基础的一个程序啦,就只是把汉字编码成二进制串。我在考虑怎么把这个串再压缩压缩。
        }
        \dialog[AP]{
            \direction{灵光一闪} 不是只够做一个终端的吗?干脆我们直接把自己的信息编码成二进制串,然后发射到太空中去!
        }
        \stage{
            沉默。
        }
        \dialog[D0]{
            \direction{打破沉默} 这个……这个有点大啊。
        }
        \dialog[D1]{
            有点大啊。
        }
        \dialog[D2]{
            有点大啊。
        }
        \dialog[D3]{
            有点大啊。
        }
        \dialog[D4]{
            有点大啊。
        }
        \dialog[AP]{
            有点……欸你们就说是不是够有意思嘛!
        }
        \stage{
            沉默。
        }
    }
    \scene{
        \stage{
            \textup{
                \textbf{时间点\(\boldsymbol{P}\)},Q中学操场。
            }
            雨淅淅沥沥地下着,夜色已深,操场上空无一人,只有几盏昏黄的路灯洒在湿漉漉的地面上。AP 思索着自己的设计方案,看着远处的天空,更加坚持自己的想法,但既不确定其是不是真的可行,也不知道如何说服同组的同学。AP 在操场上独自思考。
        }
        \stage{
            C0 从教学楼走出,看到操场上的 AP,便撑着一把伞走了过去。
        }
        \dialog[C0]{
            \direction{走到 AP 身边} 怎么,这么晚了还在这儿?\direction{递伞} 还淋着雨,会感冒的,快回去吧。
        }
        \dialog[AP]{
            \direction{抬起头,勉强挤出一个笑容,接过伞,看向远方} 我在想,我的设计方案,是不是太大了?
        }
        \dialog[C0]{
            \direction{看着远方,思考} 你说的是,关于那个,把信息编码成二进制串,然后发射到太空的想法吗?
        }
        \dialog[AP]{
            \direction{回头} 嗯。
        }
        \dialog[C0]{
            你这个想法确实比较大胆,也确实需要一些技术支持和更多的论证。但最重要的是,你要相信自己的想法,并且努力去实现它。\direction{抬头} 你有想过,这个方案的意义吗?
        }
        \dialog[AP]{
            \direction{思考} 意义?
        }
        \dialog[C0]{
            \direction{指向远方} 你看,那片天空,那些星星,那些星星背后的故事,那些故事背后的意义,你有想过吗?
        }
        \dialog[AP]{
            \direction{看向远方} 我……
        }
        \dialog[C0]{
            论证的部分,我可以帮你查,技术的部分,我也可以帮你做。但是,你要相信自己的想法,相信自己的能力,相信自己的意义。\direction{看向 AP} 因为这段时间我们终究时间有限,要实现更多的可能性,你还是需要未来自己去思考。
        }
        \dialog[AP]{
            \direction{点头} 嗯。
        }
        \dialog[C0]{
            回去吧,明天还有很多事情要做,你也需要休息。\direction{拍拍 AP 的肩膀} 别被现有的框架所限制,回去多想想那个关于「意义」的问题,你会找到答案的。
        }
        \dialog[AP]{
            \direction{看着远方,思考} 意义……
        }
        \stage{
            第二天,C0 带着 AP 和其他组员们再次聚集在机房。C0 主动提起了 AP 的想法,拿出了一夜论证的结果,并鼓励大家一起去尝试和实现。同学们也被 C0 的鼓励所感染,纷纷在这个方案的基础上提出了更多的想法,开始了他们的设计和实现。【此段不详细展开,用蒙太奇的方式表现】
        }
    }
    \scene{
        \stage{
            \textup{
                \textbf{时间点\(\boldsymbol{P}\)},Q中学操场。
            }
            结营展示当天,AP 组的作品被特别放在最后展示。前面的作品展示完毕后,大家从礼堂走上操场,AP 组的同学们在操场中央的展示区域前等待着。大家围成一个大圈,把他们围在中间。
        }
        \dialog[主持人]{
            \direction{在中央} 接下来,我们有一个特别的展示,之后就是最后的颁奖环节。下面,我们先把时间给到操场上等待着的同学们!
        }
        \stage{
            众人鼓掌。AP 带着发射装置走上展示区域,向四周的观众鞠躬。C0 也走上去,向观众们挥手致意。
        }
        \dialog[AP]{
            \direction{向观众们展示,介绍作品的设计动机、技术细节和功能特点} (暂略)
        }
        \stage{
            众人鼓掌。AP 和 C0 一起将板子和电脑连接好,将放大天线接上电源。一切准备就绪,在高涨的气氛中,AP 用摇杆将信号输入设备,传入高空。
        }
    }
}

%%%%%%%%%%%
\act[复燃]{
%%%%%%%%%%%
    \scene{
        \stage{
            \textup{
                \textbf{时间点\(\boldsymbol{C}\)},T大学电子系实验室天台。
            }
            承接第一幕,AC 回过神来,看着串口绘图仪上的图像。B0 和 B1 也看着绘图仪,但都还认为是故障,并不在意。
        }
        \dialog[B1]{
            这监视器怎么回事?是哪里的工频信号进来了吗?
        }
        \dialog[B0]{
            这个波……反正不是预期,还是哪里有问题。要不还是老办法Reset一下?
        }
        \stage{
            B0 正要伸手去按 Mega 板的 Reset 键,AC 突然(有些激动地)拦住了他。
        }
        \dialog[AC]{
            \direction{激动} 不!不要!这个波……先让它把收到的数据记着!
        }
        \stage{
            AC 快速写起了程序,凭着印象中当年的编码方式写出了一个解码程序。绘图仪中的图像慢慢平缓下来,突然出现的信号消失了,只剩下噪声起起伏伏。
        }
        \stage{
            AC 将串口数据复制出来,导入刚刚写好的解码程序,终于在屏幕上看到了一行行的汉字。
        }
        \dialog[B0、B1]{
            \direction{惊讶} 啊这……
        }
        \stage{
            AC 也有些激动,但又有些迷茫,屏幕上的汉字让他不得不相信,当年的信号不知为何有了回波。
        }
        \stage{
            AC 解释着自己的难以置信却又铁证如山的想法,感觉到了当年的那份探索热情。冥冥之中,似乎确实有什么东西在等待着他。这种意义感让他重新振作起来。
        }
    }
    \scene{
        \stage{
            \textup{
                \textbf{时间点\(\boldsymbol{C}\)},P 中学机房。
            }
            实践支队来到 P 中学的支教活动前夜,包括 AC 在内的几位同学正在一台一台地手动安装软件,为即将开始的支教活动做准备。窗外,刚下晚自习的 E 背着包、抱着书,靠在窗边向里张望。
        }
    }
}

%%%%%%%%%%%
\act[再会]{
%%%%%%%%%%%
    \scene{
        \stage{
            \textup{
                \textbf{时间点\(\boldsymbol{F}\)},T大学实验室。
            }
            AF 已学有所成,经过多次试验和改进,时空沟通的课题已经基本完成。AF 正在检查着最后的程序,准备进行最后的实验。
        }
        \dialog[AF]{
            \direction{自言自语} 这次的实验,应该是最后一次了。 \direction{看向屏幕} 一切准备就绪,开始吧。
        }
        \stage{
            AF 按下了启动键,信息流开始在时空中穿梭,汇集到他的面前。
        }
        \stage{
            手机振动,是他指导的支教队的同学发来的信息。灵光一闪,AF 想起了当年的自己。
        }
        \dialog[AF]{
            \direction{自言自语} 那么这最后的试验,就从那时的我开始吧。 \direction{开始输入时空坐标,突然明白了什么} 原来,这就是当时那列回波的来源啊。时空的沟通,原来是这样的。 \direction{看向屏幕} 那么,我要把这首诗送回去了。
        }
        \stage{
            AF 开始操作起来,连接起 时间点\(C\) 和 时间点\(P\) 的信息流。望着屏幕上的不断跳出的运行日志,感慨万千。
        }
        \dialog[AF]{
            \direction{放松下来} 把过去的我的礼物,送给过去的我,还真是有意思的事情啊。 
        }
        \dialog[AF]{
            既然这样,\direction{转向观众} 不如也将这些信息流,送到各位所处的这个特殊的时候吧。
        }
        \stage{
            AF 操作起来,时间点\(C\)、时间点\(P\) 和 时间点\(F\) 的信息流汇聚起来,形成了一个巨大的信息流。信息流在时空中穿梭,最终汇聚到了一个点上。屏幕上显示,那里正在举办 T 大学电子系的学生节,舞台暗场又亮起,屏幕上显示着「频率之外」。
        }
        \dialog[AF(画外音)]{
            希望你们,喜欢这个有意思的故事。
        }
        \stage{
            全剧终。
        }
    }
}

%-----------------------------------------------------------

%%%%%%%%%%%%%%%%%%%
\chapter*{拍摄计划}
%%%%%%%%%%%%%%%%%%%
\addcontentsline{toc}{chapter}{拍摄计划}

\section*{DEMO拍摄}
\addcontentsline{toc}{section}{DEMO拍摄}

\begin{table}[!ht]
    \raggedright
    \begin{tabular}{p{0.2\textwidth}p{0.2\textwidth}p{0.5\textwidth}}
        \toprule
        ~ & & \\[-0.7em]
        \textbf{时间} & \textbf{地点} & \textbf{内容} \\[0.3em]
        \midrule
        ~ & & \\[-0.7em]
        10月11日        & 紫荆操场 & \textbf{ACT 2, SCENE 3} \\[0.3em]
        22:00~23:59    & ~ & 
            AP 受到打击,在操场上独自思考,被 C0 发现,获得宽慰。
        \\
        ~ & & \\[-0.7em]
        ~ & ~ & \textbf{相关角色\quad}AP、C0 \\[0.3em]
        \bottomrule
    \end{tabular}
\end{table}

\begin{table}[!ht]
    \raggedright
    \begin{tabular}{p{0.2\textwidth}p{0.2\textwidth}p{0.5\textwidth}}
        \toprule
        ~ & & \\[-0.7em]
        \textbf{时间} & \textbf{地点} & \textbf{内容} \\[0.3em]
        \midrule
        ~ & & \\[-0.7em]
        10月12日        & 某个教室 & \textbf{ACT 2, SCENE 2} \\[0.3em]
        8:00~9:35      & ~ & 
            AP 及小组成员在机房讨论设计方案,C0 帮助他们调试程序。
        \\
        ~ & & \\[-0.7em]
        ~ & ~ & \textbf{相关角色\quad}AP、D0~D4、C0 \\[0.3em]
        \bottomrule
    \end{tabular}
\end{table}
{   \vspace{-0.7\baselineskip}
    \kai \small
    \textbf{注:}该场由于校服难以到位,预计正式拍摄需要重拍。
}

% %%%%%%%%%%%%%%%%%%%
% \chapter*{可用的梗}
% %%%%%%%%%%%%%%%%%%%
% \addcontentsline{toc}{chapter}{梗}

% \begin{enumerate}
%     \item 如果你有两块面包,你应该把两块面包都吃掉,因为一块面包价值五十万马克。
% \end{enumerate}

%=========================================================
\end{document}